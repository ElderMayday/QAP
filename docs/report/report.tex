\documentclass[12pt]{article}

\usepackage{caption}
\usepackage{float}
\usepackage{hyperref}
\usepackage{datetime}
\usepackage{mathtools}


% for listings
\usepackage{listings}
\lstset{
	numbers=left, 
	numberstyle=\small, 
	numbersep=8pt, 
	frame = single, 
	breaklines=true,
	tabsize=2,
	framexleftmargin=15pt
}



\author{Student \textbf{Aldar Saranov (000435170 ULB)} \\ Aldar.Saranov@ulb.ac.be}

\date{\today}

\title{Report for the project for the Swarm Intelligence course}

\begin{document}

\maketitle
\newpage

\section{Implementation}

In the literature many different methods are proposed and researched including various implementations of the simulated annealing, taboo search, hybrid genetic-taboo search. However, for this project an algorithm known as Hybrid Ant System for the Quadratic Assignment Problem (HAS-QAP) was used as it was propose by Gambardella and Dorigo in \cite{Gambardella}. As all the ACO algorithms it uses the notion of solution construction biasing by means of pheromone trails, deposited by ants.

\begin{minipage}[c]{0.95\textwidth}
\begin{lstlisting}[caption={General ACO pseudo-code}, label={lst:aco},mathescape]
procedure ACO-Metaheuristic
generate m random permutations $\pi^1$,...,$\pi^m$.
[optionally] improve $\pi^1$,...,$\pi^m$ by local search
let $pi^*$ be the best solution
initialize the pheromone trail matrix T
activate intensification
for i=1 to $I^{max}$
	for k from 1 to m
		$\hat{\pi}^k$ = PheromoneTrailSwaps($\pi^k$)
	end
end
\end{lstlisting}
\end{minipage}


\bibliographystyle{plain}
\bibliography{ref}

\end{document}

